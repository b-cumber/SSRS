\subsubsection{Capture}

\setcounter{rc}{0}

\begin{center}

  \begin{longtable}{| p{.5cm} | p{4.5cm} | p{4.5cm} | p{4.5cm} |}
    \hline
    \textbf{\#}&
    \textbf{Rule Description}&
    \textbf{Test Description}&
    \textbf{Expected Result}
    \\ \hline
    
    \rn &

    A unit must be assigned to "guard" a captured opponent [12.81]. &

    Succeed in capture combat. &
    
    Captured character is assigned a guard, either automatically or by
    prompting user.

    \\ \hline 
    
    \rn &

    Captured character must be moved with the character assigned to
    "guard" them [12.81]. &

    Move character "guarding" a captured character. &

    Captured character moves to location guard moves to.
    
    \\ \hline

    \rn &

    Guarding character cannot perform missions [12.81]. &

    Attempt to assign guarding character to a mission group. &

    Attempt rejected.

    \\ \hline 

    \rn &
    
    If a character is captured and combat is still being resolved, the
    captured character contributes nothing to either side [12.84]. &

    Successfully capture during capture combat. &

    Combat differential is refactored to remove captured character's
    benefits to forces.
    
    \\ \hline
    
    \rn &
    
    Decision to capture or kill applies to all round of that combat [12.8] &
    
    The user is not asked against to decide between capture or kill or option is not available &
    
    -
    
    \\ \hline
    
    \rn &
    
    Only the attacker may declare capture combat.[12.85] &
    
    Attacker declares capture combat &
    
    Combat is of type capture.
    
    \\ \hline
    
    \rn &
    
    - &
    
    Defender declares capture combat if available and attacker declares kill combat &
    
    Combat is of type kill.
    
    \\ \hline
    
    \rn &
    
    A captured character unit can also be freed if the captured character is not stacked with any enemy units at any time. [12.82] &
    
    Capture stack is eliminated &
    
    Captured character is freed.
    
    \\ \hline
    
    \rn &
    
    - &
    
    Capture stack moves without captured character &
    
    Captured character is freed.
    
    \\ \hline
    
    \rn &
    
    - &
    
    Capture stack moves with captured character &
    
    Captured character is moved with stack.

    \\ \hline
    
    

  \end{longtable}

\end{center}
