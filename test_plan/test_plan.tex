\documentclass[report]{article}
\usepackage{longtable}
\usepackage{tabularx}
\usepackage{graphicx}
\usepackage
[
        left=2.7cm,
        right=2.7cm,
]
{geometry}

\newcommand{\comment}[1]{{\tt #1}}

\begin{document}

\title{\textsc{Test Plan}
  \\ Freedom in the Galaxy, Python Team
  \vspace{10 mm}
  \\ Version 1.0
  \vspace{10 mm}}
  
\date{\today}

\author{Prepared for:
  \\ CS 383 Course Project
  \vspace{10 mm}
  \\Prepared by:
  \\ CS 383 Python Team
  \\ University of Idaho
  \\ Moscow, ID 83844-1010
  \vspace{10 mm}}


\maketitle
\newpage 

\begin{center}
\noindent Freedom In The Galaxy Test Plan, Python Group

\vspace{10 mm}

\noindent \textsc{Record of Changes}   

\vspace{10 mm}


\begin{tabularx}{\textwidth}{| X | X | X | X | X | X | X |}
  \hline
  \textbf{Change Number} &
    \textbf{Date Completed} &
    \textbf{Location of Change (e.g. page \# or figure \#)} &
    \textbf{Brief Description of Change} &
    \textbf{Approved by (initials)} &
    \textbf{Date Approved} 
    \\ \hline 1 & Nov 9 & ALL  & INITIAL PREP & - & -
<<<<<<< HEAD
    \\ \hline N/A & -& -& -& -&  
=======
    \\ \hline 2 & Nov 11 & Sec5.1 pg6 ; Sec6 pg6 ; Sec7 pg6 ; Sec8.5 pg6 ; Sec9 pg8-9 ; Sec9.2.1 pg9-13 & 1st Draft Review and Approval & EThom JHall & Nov 11  
>>>>>>> f48c05a3a81d528f5fafa9566de8530f216836cd
%    \\ \hline & & & & &  
%    \\ \hline & & & & &  
%    \\ \hline & & & & &  
    \\ \hline
\end{tabularx}
\end{center}
\newpage 
\tableofcontents
\newpage

\section[IDENTIFIER]{TEST PLAN IDENTIFIER}
FREEDOM IN THE GALAXY MASTER TEST PLAN VERSION 1.0, PYTHON GROUP
\section[REFERENCES]{REFERENCES}
Use Cases and State diagrams are available at {\tt https://github.com/Freedom-Galaxy}.


\section[INTRODUCTION]{INTRODUCTION}
The purpose of this master test plan is to state the processes used by the Python Team in testing the Freedom in the Galaxy software project. This master plan covers the entire testing plan for the project. The Python Team is divided into three sub teams. Each sub team each has different testing requirements and should prepare documents accordingly.

\section[TEST ITEMS]{TEST ITEMS}
\begin{itemize}
\item Class Interfaces
\item Class Interactions
\item User Interface Functions
\item Network Layer Functions
\item API Level Functions
\item Requirements stated in System Software Requirement Specification
\item Requirements stated in System Software Design Documentation
\end{itemize}

\section[SOFTWARE RISK ISSUES]{SOFTWARE RISK ISSUES}
\label{risk}
Each piece of code has different levels of risk to the party. Standard library modules are usually designed to be backwards compatible, do not change for each version of Python, are heavily tested before incorporation into the standard library, and have a high level of use. Therefore, the standard library modules have a very low level of risk to the project. The tests will therefore be focused on the correct use of those modules.

Third party software has a higher level of risk than standard library and a much larger variance of quality. Some of the software has been heavily tested and some is only in a prerelease state. The level of quality of each third party module will have to be determined and testing scaled according. The level of risk can be reduced if sufficient tests are included with the library.

Testing will mainly focus on code written by Python Group members. This plan largely focuses on providing a testing plan that adequately covers both rules and implementation.

As the code base is not yet complete, the complexity of the code cannot be yet determined. Also, since the documentation is often incomplete, risk also arises because of poor documentation. Complete and improved documentation of all areas of code with reduce this risk.

Finally, because the length of this project is short and is not dependent on any third party schedules, there is no to little risk from new versions of software or failure of any third party.

\subsection{Critical Risk Areas}
\begin{enumerate}
<<<<<<< HEAD
\item Whatever the networking interface module is called
=======
\item RPyc
>>>>>>> f48c05a3a81d528f5fafa9566de8530f216836cd
\item PyGame
\item SQLAcademy
\end{enumerate} 

\section[FEATURES TO BE TESTED]{FEATURES TO BE TESTED}
\begin{center}
\begin{tabularx}{\textwidth}{| X | c |}
  \hline
  \textbf{Feature Description} &
    \textbf{Risk Level} 
\\ \hline
Start a game & High
\\ \hline
<<<<<<< HEAD
Smart AI & Medium
\\ \hline
Network communication & High
\\ \hline
Play game with graphical interface & High
\\ \hline
Play game with text interface? & Low
\\ \hline
Other features that people want to be able to do? & -
\\ \hline
From a users point of view. & -
=======
Smart AI & Low
\\ \hline
Network communication & High
\\ \hline
Play game with graphical interface & Med
\\ \hline
Play game with text interface? & Low
\\ \hline
User Experience & -
>>>>>>> f48c05a3a81d528f5fafa9566de8530f216836cd
\\ \hline
\end{tabularx}
\end{center}

\section[FEATURES NOT TO BE TESTED]{FEATURES NOT TO BE TESTED}
\begin{center}
\begin{tabularx}{\textwidth}{| X | X |}
  \hline
  \textbf{Feature Description} &
    \textbf{Details} 
\\ \hline
<<<<<<< HEAD
 Python Standard Library
 \\ \hline
 Probably we can put third party modules here.
 \\ \hline
 Speed of software?
\\ \hline



=======
 Python Standard Library.
 \\ \hline
 Probably we can put third party modules here.
 \\ \hline
 Speed of software.
\\ \hline

>>>>>>> f48c05a3a81d528f5fafa9566de8530f216836cd
\end{tabularx}
\end{center}


\section[APPROACH]{APPROACH}
\subsection{Testing Tools}
The testing tools for use in this project include the standard library modules doctests, unittest, and the third party tool Nose. Each tool has different levels of complexity and strengths, so each test will be written in one of these tools depending on the tester and condition.

Training will be available from other members of the team for any questions. Tutorials are also available on the Python website for doctest and unittest. Training for Nose is unknown.

\subsection{Metric}
The number of game rules implemented will be the metric.

\subsection{Configurations}
The AI and UI will be tested using both the Rebel and Imperial player, and all User type tests for both player types will be executed.

\subsection{Software}
<<<<<<< HEAD
The software will be developed with Python 2.7.5 with doctest and unittest of the same version. The current version of Nose is 1.3.0, the current version of Pygame is ?
\subsection{Hardware}
\begin{itemize}
\item Microsoft Windows XP Professional SP3/Vista SP1/Windows 7 Professional
\begin{itemize}
	\item Processor: 800MHz Intel Pentium III or equivalent
	\item Memory: 512 MB
	\item Disk space: 750 MB of free disk space
\end{itemize}
\item Ubuntu 9.10
\begin{itemize}
	\item Processor: 800MHz Intel Pentium III or equivalent
	\item Memory: 512 MB
	\item Disk space: 650 MB of free disk space
\end{itemize}
\item Macintosh OS X 10.6 Intel
\begin{itemize}
	\item Processor: Dual-Core Intel (32 or 64-bit)
	\item Memory: 1 GB
	\item Disk space: 650 MB of free disk space
\end{itemize}
\end{itemize}

=======
The software will be developed with Python 2.7.5 with doctest and unittest of the same version. The current version of Nose is 1.3.0, the current version of Pygame is 1.2.1.
\subsection{Hardware}
The harware specifications used for testing will be at or higher than the minimum hardware specifications for Python 2.7.5.

%\begin{itemize}
%\item Microsoft Windows XP Professional SP3/Vista SP1/Windows 7 Professional
%\begin{itemize}
%	\item Processor: 800MHz Intel Pentium III or equivalent
%	\item Memory: 512 MB
%	\item Disk space: 750 MB of free disk space
%\end{itemize}
%\item Ubuntu 9.10
%\begin{itemize}
%	\item Processor: 800MHz Intel Pentium III or equivalent
%	\item Memory: 512 MB
%	\item Disk space: 650 MB of free disk space
%\end{itemize}
%\item Macintosh OS X 10.6 Intel
%\begin{itemize}
%	\item Processor: Dual-Core Intel (32 or 64-bit)
%	\item Memory: 1 GB
%	\item Disk space: 650 MB of free disk space
%\end{itemize}
%\end{itemize}
%\end{comment}
>>>>>>> f48c05a3a81d528f5fafa9566de8530f216836cd
\subsection{Automated Testing}
The automated test process will be divided into three phases, Unit testing, Integration testing and System testing.

\subsubsection{Unit Testing}
Unit testing will be used to validate individual classes objects. Some objects will not be required to be validated by unit testing, such as the user interface and the supporting classes and any classes or code that is exempted according to section \ref{risk}.

Each team should develop TCSs as needed for their code. Each TCS should include tests for all classes as required and for each test, validate as many inputs as needed to test that the class operates as required by the rules covered in the SSRS, and also according to an requirements and design documents applicable to that class.

\subsubsection{Required Class Unit Tests}
\begin{itemize}
\item What classes do we have? Each class should have tests?
\end{itemize}
Unit testing for each of these classes will consist of validating the class public interface. Public variables and methods are required to pass the following tests:
\begin{itemize}
\item validation using expected data input
\item validation using erroneous data input, including None or empty values
\item class instantiation
\end{itemize}

\subsubsection{Integration Testing}
Integration testing will validate the defined associations between classes. This will be an incremental process performed as new classes are created and after classes have successfully passed their unit testing phase. 
\paragraph{Integration Tests to be Performed}
\begin{itemize}
\item Class sets under test exhibit the relationships as defined in the class diagram
\item Methods and functions are called using the correct parameters
\item Methods and functions return the expected data types or structures.
\end{itemize}

Integration should be done at the group level. This should include mock objects or mock data to simulate any objects outside of the team as required. Integration testing should cover all reasonable combinations of objects in the game. These tests should be documented in TCSs.

\subsubsection{System Testing}
System testing will exercise the overall software product to verify conformance to defined game rules. System testing will use predefined scenarios of initial states representing possible game states that a user will encounter. Each scenario will be tested for acceptable results for each possible user action that may occur. Acceptability will be defined by the game rules. 

\subsection{Manual Testing}
Manual testing will be required for the portions of the program that can not undergo automated testing. This section applies to the testing of the user interface to simulate user orientated testing to verify conformance to the documented use cases.


\section[ITEM PASS/FAIL CRITERIA]{ITEM PASS/FAIL CRITERIA}

\subsection{Reporting a failure}

If a failure happens during the execution of any test, a failure report should be submitted to provide information of the failure. Failure should be reported in the issue page of the repository. 

The name of the issue should be the name of the name of the code file and the name of the test if applicable.

\subsubsection{What Constitutes a "failure"?}
Any deviation from a specification, e.g. SRS, UML diagrams.

\subsubsection{What Does Not Constitute a "failure"?}
Any unit or action that does not have any requirements documentation, cannot cause a 'failure'.

\subsubsection{What to do when a "failure" is discovered"?}
Produce a SCR to document each "failure" that needs to be corrected.

\subsubsection{What if a specification document is incorrect (e.g. outdated, misstated)?}
This also constitutes a "failure" and an SCR should be created.

\subsubsection{What sections to include in an SCR}

<<<<<<< HEAD
Problem \\
What did you expect to happen \\
What happened \\
=======
Failure Identified\\
Expected Outcome \\
Actual Behavior \\
>>>>>>> f48c05a3a81d528f5fafa9566de8530f216836cd
Steps to reproduce \\

Each section should be brief and to the point, but yet convey enough information for the coder.

\subsection{Unit Testing Pass/Fail Criteria}
Each of these tables might need to be expanded to cover priority and type of test (i.e. unittest, integration, system). Also, additions of conditionals might also be good.

<<<<<<< HEAD
\input{character_unit_tests}
=======
%\input{character_unit_tests}
>>>>>>> f48c05a3a81d528f5fafa9566de8530f216836cd

\newpage

\subsection{Integration Testing Pass/Fail Criteria}

<<<<<<< HEAD
   \subsubsection{Game Turn}

For each set of tests, test both Rebel players and Imperial players.

\setcounter{rc}{0}

\begin{center}

  \begin{longtable}{| p{.5cm} | p{4.5cm} | p{4.5cm} | p{4.5cm} |}
    \hline
    \textbf{\#}&
    \textbf{Rule Description}&
    \textbf{Test Description}&
    \textbf{Expected Result}
    \\ \hline
    
    \rn &
    
    Mission Assignment only occurs on the player's Mission Assignment Segment &

    Start player Mission Assignment during player's Mission Assignment Segment &
    
    Mission Assignment Accepted. 
    
    \\ \hline 
    
    \rn &
     - &
     
    Start non-phasing player Mission Assignment during player's Mission Assignment Segment &
    
    Mission Assignment Rejected.
    
    \\ \hline 
    
    \rn &
    
    - &
    
    Start player's Mission Assignment during player's Mission Action Segment &
    
    Mission Assignment Rejected.
    
    \\ \hline 
    
    4 &
    
    - &
    
    Start player's  Mission Assignment during player's Operations Phase &
    
    Mission Assignment Rejected.

    \\ \hline 
    
    5 &
    
    - &
    
    Start player's Mission Assignment during player's Mission Phase &
    
    Mission Assignment Rejected
    
    \\ \hline 
    
    6 &
    
    - &
    
    Start Rebel Mission Assignment during the Imperial's Operations Phase &
    
    Mission Assignment Rejected
    
    \\ \hline 
    
    7 &
    
    - &
    
    Start Rebel Mission Assignment during the Imperial's Search Phase &
    
    Mission Assignment Rejected
    
    \\ \hline 
    
    8 &
    
    - &
    
    Start Rebel Mission Assignment during the Game Turn Interphase &
    
    Mission Assignment Rejected
    
    \\ \hline 
    
    9 & 
    
    During a player's Movement Segment, only the phasing player's units can move &
    
    Move movable unit of phasing player &
    
    Movement Accepted
    
    \\ \hline 
    
    10 &
    
    - &
    
    Move movable unit of non-phasing player &
    
    Movement Rejected
    
    \\ \hline 
    
    11 &
    
    During a player's Reaction Segment, only the non-phasing units can move &
    
    Move movable unit of non-phasing player according to reaction rules during Reaction Segment &
    
    Movement Accepted
    
    \\ \hline 
    
    12 &
    
    - &
    
    Move movable unit of phasing player according to reaction rules &
    
    Movement Rejected
    
    \\ \hline 
    
    13 &
    
    - &
    
    Move moveable unit of non-phasing player according to reaction rules not during Reaction Segment &
    
    Movement Rejected
    
    \\ \hline 

  \end{longtable}

\end{center}

\newpage
\subsection{Capture}

\setcounter{rc}{0}

\begin{center}

  \begin{longtable}{| p{.5cm} | p{4.5cm} | p{4.5cm} | p{4.5cm} |}
    \hline
    \textbf{\#}&
    \textbf{Rule Description}&
    \textbf{Test Description}&
    \textbf{Expected Result}
    \\ \hline
    
    \rn &

    A unit must be assigned to "guard" a captured opponent [12.81]. &

    Succeed in capure combat. &
    
    Captured character is assgned a guard, either automatically or by
    prompthing user.

    \\ \hline 
    \rn &

    Captured character must be moved with the character assigned to
    "guard" them [12.81]. &

    Move character "guarding" a captured character. &

    Captured character moves to location guard moves to.
    
    \\ \hline

    \rn &

    Guarding character cannot perform missions [12.81]. &

    Attempt to assign guard to mission group. &

    Guard does not occupy mission group.

    \\ \hline 

    \rn &
    
    If a character is captured and combat is still being resolved, the
    captured character contributes nothing to either side [12.84]. &

    Successfully capture during capture combat. &

    Combat differential is refactored to remove captured character's
    benefits to forces.

    \\ \hline

  \end{longtable}

\end{center}

\newpage
\subsubsection{Combat}

\setcounter{rc}{0}

\begin{center}

  \begin{longtable}{| p{.5cm} | p{4.5cm} | p{4.5cm} | p{4.5cm} |}
    \hline
    \textbf{\#}&
    \textbf{Rule Description}&
    \textbf{Test Description}&
    \textbf{Expected Result}
    \\ \hline

    \rn &

    If a character stack in an environ is detected and there are no
    friendly military forces present, if they successfully break off
    from combat, they are no longer found and are no longer a part of
    combat [12.51]. &

    Break off from combat. &

    Character state is no longer found and combat is terminated.

    \\ \hline

    \rn & 

    If a character receives cumulative damage equal to their number of
    endurance points that character is dead and is removed from play
    [12.73]. &

    Engage in combat that results in enough wounds to match a
    character's endurance and assign those wounds to that character. &
    
    Character is removed from play.

    \\ \hline

    \rn &
    
    If attacking group is a squad, combat is firefight [12.31]. &

    Attack with squad and check combat type.&
    
    Combat type is firefight. 

    \\ \hline
    
    \rn & 
    
    Inactive characters do not contribute to combat rating or suffer
    damage [12.42]. &

    Engage in combat and set one character to inactive. &
    
    Character's stats do not contribute to combat rating and cannot
    have wounds assigned to them after battle.

    \\ \hline

    \rn & 
    
    Attacking character does not have inactive forces [12.44]. &

    Attack enemy forces. &

    No option to set inactive forces.

    \\ \hline

    \rn & 
    
    Leaders are not eliminated in military combat but if their forces
    are destroyed they are attacked by a squad [10.45]. &

    Engage in military combat with a leader where all forces are
    destroyed. &

    Leader is not destroyed but is attacked by a squad. 

    \\ \hline
    
    \rn &
    
    Characters stacked with military units are not effected by
    military combat and do not effect military combat but if their
    military units are eliminated, they are attacked by a squad
    [10.6]. &

    Engage in military combat with a stack containing characters where
    all military forces are eliminated. &

    Characters do not effect military combat and are not eliminated
    but are engaged in squad combat.

    \\ \hline

  \end{longtable}

\end{center}

\newpage
\subsubsection{Game Start}
\begin{center}

  \begin{longtable}{| p{.5cm} | p{4.5cm} | p{4.5cm} | p{4.5cm} |}
    \hline
    \textbf{\#}&
    \textbf{Rule Description}&
    \textbf{Test Description}&
    \textbf{Expected Result}
    \\ \hline
    
    1 &

    At game start, if player controls a planet, they may distribute
    characters and military forces on environs of those planets
    [16.12] &

    Player attempts to place a unit on the environ of a planet the
    player does not control &

    Unit does not occupy specified environ.  ??? Fail silently or
    notify player ???

    \\ \hline

    2 &

    At game start, if player does not control a planet, characters
    will "arrive from space" at the beginning of their operations
    phase. [16.12] &

    Player attempts to place units on environ of a planet they do not
    control &

    Unit does not occupy specified environ.  ??? Fail silently or
    notify player ???

    \\ \hline 

    3 &

    At game start, Imperial player places units first [16.14] &

    Rebel player attempts to place units \textit{before} Imperial
    player &

    Rebel player is unable to place forces.

    \\ \hline

    4 &

    At game start, Imperial player places units first [16.14] &

    Rebel player attempts to place units \textit{while} Imperial
    player \ is distributing their forces &

    Rebel player is unable to place forces.
    
    \\ \hline

    5 &
    Rebel game turn is first [16.14] &

    At start of game Imperial player attempts to begin turn
    \textit{before} Rebel player had begun turn &
    
    Imperial player is unable to begin turn.

    \\ \hline 

    6 &
    Rebel game turn is first [16.14] &

    At start of game Imperial player attempts to begin turn
    \textit{while} Rebel player is in their turn &

    Imperial player is unable to begin turn.

    \\ \hline 

    7 &

    At start of game, all friendly units on the same environ are
    placed in the same stack [16.13]. &

    Try and create a second stack in an environ at the beginning of
    play &

    Second stack is not constructed.

    \\ \hline

    8 &

    At start of game, if rebel player has units to distribute, units
    placed must match environ type [16.13]. &

    Attempt to place rebel units in environ that does not match their
    type. &

    User unable to place forces.

    \\ \hline 

  \end{longtable}

\end{center}

\newpage
\subsubsection{Missions}

\setcounter{rc}{0}

\begin{center}

  \begin{longtable}{| p{.5cm} | p{4.5cm} | p{4.5cm} | p{4.5cm} |}
    \hline
    \textbf{\#}&
    \textbf{Rule Description}&
    \textbf{Test Description}&
    \textbf{Expected Result}
    \\ \hline
    
    \rn &

    Diplomacy missions by a rebel player on a patriotic planet have 2
    bonus draws subtracted [13.2]. &

    Undergo diplomacy mission on a patriotic planet. &
    
    Total bonus draws are reduced by two. 

    \\ \hline 
    
    \rn &

    Diplomacy missions by a rebel player on a planet in dissent have 1
    bonus draw added [13.2]. &

    Undergo diplomacy mission on a planet in descent. &
    
    Total bonus draws are increased by one. 

    \\ \hline

    \rn &

    Subvert troops, start rebel camp and scavenge missions can only be
    performed by the rebel player [13.2]. &

    During the missions phase as Imperial player, view available
    missions. &

    Scavenge, start rebel camp and subvert troops missions are not
    available options.

    \\ \hline 
    \rn &

    Diplomacy missions by a rebel player on a patriotic planet have 2
    bonus draws subtracted [13.2]. &

    Undergo diplomacy mission on a patriotic planet. &
    
    Total bonus draws are reduced by two. 

    \\ \hline 

    \rn &
    
    A rebel camp is equivalent to a mission group for mission purposes
    [13.2]. &

    Assign rebel camp a mission in an environ during the mission
    phase. &

    Rebel is assigned mission. 

    \\ \hline

    \rn &
    
    Rebel camps cannot receive bonus draws [13.2]. &

    Draw action card resulting in a bonus draw with only a rebel camp
    doing missions in an environ. &

    No bonus draws are awarded. 

    \\ \hline 

    \rn &
    
    Rebel camps are never effected by the effects of an action card. &

    Perform mission with rebel camp and draw an action card resulting
    in combat. &

    Rebel camp does not engage in combat. 
 
    \\ \hline

    \rn &

    There can never be more than one rebel camp in an environ [13.2]. &

    Attempt start rebel camp mission in an environ with a rebel camp. &

    Mission is not available. 

    \\ \hline 

    \rn &
    
    Rebel camps cannot be moved [13.2]. &

    View movable units in an environ with a rebel camp during
    operations phase. &

    Rebel camp is not displayed. 

    \\ \hline 

    \rn &
    
    Characters cannot accompany rebel camps on a mission [13.2]. &

    Break mission stack into mission groups and attempt to add a
    character to a rebel camp mission group. &
    
    Character does not occupy rebel camp's mission group. 
    
    \\ \hline 

    \rn &

    Spaceship quest, summon sovereign, question prisoner and steal
    resources missions are not available in the star system games
    [13.2]. &

    View available missions during missions phase of a star system
    game.&
    
    Spaceship quest, summon sovereign, question prisoner and steal
    resources missions are not available.
    
    \\ \hline 

    \rn &

    Maximum number of action cards drawn for mission groups in an
    environ is equal to the environ size [13.3]. &

    Draw four action cards in an environ of size four. &

    Only bonus draws remain.

    \\ \hline  

    \rn &

    Bonus draws are only available is mission group has drawn maximum
    number of action cards for the environ size [13.3].&

    Quit missions in an environ of size 4 after drawing 1 action card
    when a character has bonus draws for the mission type. &

    No bonus draws available. 
    
    \\ \hline

    \rn &

    Action card must be resolved before mission letter can take effect
    [13.3]. &

    Die in combat with creatures resulting from an action card that
    would have otherwise resulted in mission success. &
    
    Mission does not succeed. 
    
    \\ \hline

    \rn &

    Except for the case of the NPP searching for characters on a mission, if
    an action drawn is the same as an action previously drawn in that
    environ, the action is ignored [13.42]. &

    Draw an action for creature combat after creature combat has
    already been resolved during that turn in that environ. &

    Mission group does not engage in creature combat. 

    \\ \hline

    \rn &

    In the case of an action card that allows the NPP to search for
    characters on a mission, this action may be repeated \textit{n}
    times where \textit{n} is the number of mission groups in that
    environ [13.48].  &

    Draw action card for two mission groups resulting in NPP search
    after NPP has already searched once in the environ.  &

    NPP performs search for mission groups in current mission environ.

    \\ \hline

    \rn &

    In the case of an action card that allows the NPP to search for
    characters on a mission, this action may be repeated \textit{n}
    times where \textit{n} is the number of mission groups in that
    environ [13.48].  &

    Draw action card for two mission groups resulting in NPP search
    after NPP has already searched twice in the environ. &

    NPP does not perform a search.    
    
    \\ \hline

    \rn &

    If an action card is drawn that contradicts an action drawn
    previously for a mission group in an environ in a turn, the
    contradictory action is ignored [13.42]. &

    Draw an action card that prohibits bonus draws after drawing an
    action card that grants one extra bouns draw. &

    Extra bonus draw is kept, bonus draws are not prohibited. 

    \\ \hline
    
    \rn &

    If a single mission group is affected by an action card, that
    mission group is randomly chosen from all active mission groups in
    that environ [13.43]. &

    Draw action card resulting in creature combat in an environ
    containing two mission groups. &

    One of the two mission groups is chosen at random to engage in
    creature combat.
    
    \\ \hline

    \rn &

    If no creature is named by an environ the PP is attacked by sentry
    robots only if the NPP controls the planet [13.46]. &

    Draw creature attack in an environ where no creature is named and
    the NPP controls the planet. &

    Player engages in combat with sentry robots. 

    \\ \hline

    \rn &

    If no creature is named by an environ the PP is attacked by sentry
    robots only if the NPP controls the planet [13.46]. &

    Draw creature attack in an environ where no creature is named and
    the NPP does not control the planet. &

    Player does not engage in combat.
    
    \\ \hline
    
    \rn &

    If a planet is in a state of rebellion or in rebel control, the
    rebel player ignores irate locals attacks [13.47]. &

    As the rebel player performing a mission on a planet under rebel
    control, draw action card resulting in irate locals attack. &
    
    Player does not engage in combat. 
    
    \\ \hline

    \rn &

    If a planet is in a state of rebellion or in rebel control, the
    rebel player ignores irate locals attacks [13.47]. &

    As the rebel player performing a mission on a planet in rebellion,
    draw action card resulting in irate locals attack. &
    
    Player does not engage in combat. 
    
    \\ \hline
    
    \rn &

    If an action card is drawn that results in characters being found
    and there are NPP forces in that mission environ, the NPP's forces
    must attack one mission group. [12.15] &
    
    Draw an action card resulting in characters found in an environ
    where there are enemy fores. &

    Combat is initiated by either enemy characters or enemy squad at
    NPPs choice.

    \\ \hline
    
    \rn &

    If an action card is drawn that results in characters being found
    and there are no NPP forces in that mission environ, characters
    are not detected. [12.15] &
    
    Draw an action card resulting in characters found in an environ
    where there are no enemy fores. &

    Characters are not found. 

    \\ \hline

    \rn &

    Gather information mission cannot be performed on a planet under
    control of PP is there are NPP characters or military units in the
    mission environ [15.52]. &

    View available missions on a planet under PP control in an environ
    where NPP military or detected character forces are present &

    No gather information mission available. 

    \\ \hline

    \rn &

    Coup and diplomacy missions cannot be performed on a planet in a
    state of rebellion or under rebel control [15.57]. &

    View available missions on a planet in rebellion or under rebel
    control &

    No coup or diplomacy missions available. 

    \\ \hline

    \rn &

    If a planet's state is "rebellion stopped," the loyalty marker and
    the rebel-control marker are moved together [15.75]. &

    Complete mission that results in shifting loyalty marker. & 

    Rebel-control marker shifts with loyalty marker. 

    \\ \hline

    \rn &
    
    A coup mission requires a character with an intelligence rating of
    at least 1 [13.2]. &

    Attempt to assign a coup mission to a mission group without a
    character with an intelligence rating of at least 1. &

    Mission is not assigned to mission group. 

    \\ \hline

  \end{longtable}

\end{center}

\newpage
\subsubsection{Movement}

\setcounter{rc}{0}

\begin{center}

  \begin{longtable}{| p{.5cm} | p{4.5cm} | p{4.5cm} | p{4.5cm} |}
    \hline
    \textbf{\#}&
    \textbf{Rule Description}&
    \textbf{Test Description}&
    \textbf{Expected Result}
    \\ \hline
    
    \rn &

    Characters can only move from planet to planet using a spaceship
    [9.0]. &

    Attempt to move characters without a spaceship stacked with mobile
    military units. &

    Characters are left on the environ.

    \\ \hline 
    \rn &

    Every unit/character can be moved at most once during operations
    phase [9.0]. &

    Attempt to move a unit/character after moving it once. &

    Unit/character does not move.
    
    \\ \hline

    \rn &

    Non-phasing player may make a reactin move only once the phasing
    player is finished with their operations phase [9.0]. &

    Attempt to move NPP units before PP is done with operations
    phase. &

    NPP units do not move. 

    \\ \hline 

    \rn &

    Non-phasing player may make a reactin move only once the phasing
    player is finished with their operations phase [9.0]. &

    Attempt to move NPP units during PP's operations phase. &

    NPP units do not move.     
    
    \\ \hline

    \rn &
    
    Military units cannnot move to an environ that is populated by a
    number of military units equal to it's size [9.5]. &

    Attempt to move a stack of one militart unit to an environ
    populated by a max number of military units.&

    Military unit does not occupy environ.

    \\ \hline 

    \rn &
    
    A spaceship cannot carry more characters than it's capacity stat
    dictates [9.56]. &

    Construct a character stack with a spaceship that contains more
    characters than the capacity of the spaceship dictates. Attempt to
    move stack to another environ. &

    Only the number of characters equaling the capacity of the
    spaceship move.
    
    \\ \hline

    \rn &
    
    The non-phasing player may make a move for each evviron containing
    PP military and detected forces [9.6]. &

    Enter reaction move phase when there are \textit{n} environs
    containing military units and detected forces. &

    Non-phasing player is allowed \textit{n} moves. 

    \\ \hline

    \rn &
    
    A reaction move cannot be made from one planet to another
    [9.61]. &

    Attempt to move a stack from planet to planet for a reaction
    move. &
    
    Stack does not occupy new environ.
    
    \\ \hline
    
    \rn &

    &

    &

    


    \\ \hline

  \end{longtable}

\end{center}

\newpage
\subsubsection{PDBs and Detection}

\setcounter{rc}{0}

\begin{center}

  \begin{longtable}{| p{.5cm} | p{4.5cm} | p{4.5cm} | p{4.5cm} |}
    \hline
    \textbf{\#}&
    \textbf{Rule Description}&
    \textbf{Test Description}&
    \textbf{Expected Result}
    \\ \hline
    
    \rn &

    Every planet has a PDB [8.0]. &

    Initialize galaxy. &
    
    Every planet has a PDB. 

    \\ \hline 
    \rn &

    The level of a PDB is  0, 1 or 2 [8.0]. &

    Initialize galaxy. &

    PDB levels conform to states 0, 1 or 2
    
    \\ \hline

    \rn &

    PDBs are either up or down [8.0]. &

    Initialize Galaxy. &

    PDBs conform to states up or down.

    \\ \hline 

    \rn &
    
    In Star System Games, the level of a PDB cannot be improved
    [8.13]. &

    ( ???? ) Succeed in a mission ( ???? ) that would improve PDB
    level. &
    
    PDB level does not improve.
    

    \\ \hline

    \rn &
    
    PDB can only be used if its status is "up" [8.2]. &

    Move characters from an environ on a planet where the PDB is down
    in a spaceship. &

    Characters do not undergo detection routine. 

    \\ \hline 

    \rn &
    
    If characters move across a planet without a spaceship they do not
    undergo the detection routine [9.0]. &

    Move character stack from one environ to another on the same
    planet. &
    
    Characters do not undergo detection routine. 
 
    \\ \hline

    \rn &

    If characters move across a planet with a spaceship may undergo a
    detection routine [9.0]. &

    Move character stack from one environ to another on a planet with
    a PDB up and at level 1 or higher .&

    Character stack undergoes detection routine. 

    \\ \hline 

    \rn &
    
    If a PDB incurs a loss of 2 when attacking it is placed in the
    "down" state [9.25]. &

    Incur a loss of 2 with an attacking PDB. &

    PDB state is "down."
 
    \\ \hline

    \rn &

    If a PDB incurs a loss of 3 when attacking it is placed in the
    "down" state \textit{and} it is reduced a level [9.25]. &

    Incur a loss of 3 with an attacking PDB. &

    PDB state is "down" \textit{and} level is decremented by 1.

    \\ \hline 

    \rn &
    
    A loss of 1 incurred by a PDB does nothing [9.25]. &

    Incur a loss of 1 with an attacking PDB. &
    
    PDB does not change state. 
 
    \\ \hline

    \rn &

    If a spacehip leaves a planet unable to conduct a detection
    routine, it and its characters are no longer detected [9.45]. &

    Move detected character stack in a spaceship from one environ to
    another on a planet unable to conduct a detection routine. &

    Characters are no longer detected. 

    \\ \hline 

    \rn &

    Stacks with military units are detected. &

    Move military units into a stack of undetected characters in an
    environ. &

    Characters are detected. 

    \\ \hline 

    \rn &

    If a detected character stack leaves an environ "undetected" by
    the enemy PDB they are no longer detected [9.31]. &
    
    Move detected character stack in a spaceship from an envrion on a
    planet with an active PDB where the detection routine results in
    "undetected". &

    Characters are no longer detected. 
    Character stack is no longer detected. 

    \\ \hline 

    \rn &

    A PDB with a level 0 only detects regardless of the outcome of the
    detection routine [9.32]. &

    Undergo detection routine with a PDB at level 0 resulting in
    "eliminated." &

    Character stack is detected.

    \\ \hline 

    \rn & 

    Detected characters moving from one environ to another
    containing no detected characters or military units on the same
    planet without a spaceship are no longer detected [9.35]. & 

    Move detected characters to another environ on the same planet
    that contains no detected characters and no military units. &
    
    Characters are no longer detected. 

    \\ \hline 

    \rn &

    Military units are always detected but can be attacked by a PDB. &

    Move military units from space into a planet's environ. &

    Military units undergo detection routine and are subject to attack
    accoring to the military combat results table.
    
    \\ \hline 

    \rn &

    If a spaceship is "detected and damaged" twice in the same turn,
    the second "detected and damaged" outcome is equivalent to
    "eliminated" outcome [9.35]. &

    Move from one environ to another on a planet with an active PDB at
    level 1 or more and incur "detected and damaged" outcome twice. &

    Characters and spaceship are eliminated. 

    \\ \hline 

    \rn &

    In the star system games, if a detection routine results in
    "detected and damaged," the spaceship is eliminated when the
    destination is reached [9.31]. &

    Undergo detection routine that results in "detected and damaged" &
    
    Spaceship is eliminated on arrival. 

    \\ \hline 

    \rn &
    
    If a detection routine results in "eliminated," the spaceship and
    all characters on board are eliminated immediately [9.31]. &
    
    Undergo detection routine that results in "eliminated" &

    Spaceship and characters in it are eliminated. 
    
    \\ \hline 

  \end{longtable}

\end{center}

\newpage
\subsubsection{Planets and Control}

\setcounter{rc}{0}

\begin{center}

  \begin{longtable}{| p{.5cm} | p{4.5cm} | p{4.5cm} | p{4.5cm} |}
    \hline
    \textbf{\#}&
    \textbf{Rule Description}&
    \textbf{Test Description}&
    \textbf{Expected Result}
    \\ \hline
     
    \rn &

    At the beginning of play, every planet has a loyalty counter in 1
    of five states [15.1] &

    Check every planet's loyalty after game starts. &
    
    Loyalty counters are not null.  

    \\ \hline 
    \rn &

    A loyalty counter at patriotic cannot be increased further in the
    Imperial player's favor [15.13]. &
    
    Successful Imperial "Diplomacy" or "Political Tract" mission when
    loyalty counter is set to patriotic. &

    Loyalty counter does not change state. 
    
    \\ \hline

    \rn &

    A loyalty counter at unrest cannot be increased further in the
    Rebel player's favor [15.13]. &
    
    Successful Rebel "Diplomacy" or "Political Tract" mission when
    loyalty counter is set to patriotic. &

    Loyalty counter does not change state. 
   
    \\ \hline

    \rn &
    
    Loyalty counter on a planet in a state of rebellion or rebel
    control cannot be moved [15.14]. &

    Successful Rebel "Diplomacy" or "Political Tract" mission when
    planet is in a state of rebellion or rebel control. &

    Loyalty counter does not change state. 

    \\ \hline

    \rn &
    
    A planet in a state of rebellion is controlled by neither player
    [15.4]. &

    Put a planet in the state of rebellion. &

    "Control indicator" is in neither a rebel or imperial state.

    \\ \hline 

  \end{longtable}

\end{center}

\newpage
\subsubsection{Searching}

\setcounter{rc}{0}

\begin{center}

  \begin{longtable}{| p{.5cm} | p{4.5cm} | p{4.5cm} | p{4.5cm} |}
    \hline
    \textbf{\#}&
    \textbf{Rule Description}&
    \textbf{Test Description}&
    \textbf{Expected Result}
    \\ \hline
    
    \rn &
    
    Search is conducted with either military units or
    characters. Characters stacked with military units cannot be used
    to conduct character search unless they are moved from the
    military stack [11.21]. &

    Attempt to perform character search with character stacked with
    military. &

    ??? Either search is not conducted or a military search is
    conducted instead. ???

    \\ \hline 

    \rn &

    Only the successful search group attacks [11.23].&

    Successfully search with characters. & 

    Only character combat should be initiated.
    
    \\ \hline

    \rn &

    Using characters to search results in their detection [11.31]. &

    Search with undetected characters. &11

    Characters become detected. 

    \\ \hline 

    \rn &
    
    Searching may only be done in environs where a player has units
    and there are \textit{detected} enemy forces [11.1].&

    Attempt to search an environ where there are no detected forces. &

    No search is conducted. 

    \\ \hline 
  \end{longtable}

\end{center}

\newpage
\subsubsection{Stacking}

\setcounter{rc}{0}

\begin{center}

  \begin{longtable}{| p{.5cm} | p{4.5cm} | p{4.5cm} | p{4.5cm} |}
    \hline
    \textbf{\#}&
    \textbf{Rule Description}&
    \textbf{Test Description}&
    \textbf{Expected Result}
    \\ \hline
    
    \rn &

    Number of military units in a stack, in an environ can never
    exceed the environ size [9.5] &

    Try and place a military unit in an environ that is at capacity &
    
    Military unit is not added to the stack.

    \\ \hline 
    \rn &

    There is no limit to the number of units in a stack [9.5]. &
    
    Construct an artificial stack out of all military units + all
    characters + all ships&
    
    All units, characters, ships occupy the same stack.
    
    \\ \hline

    \rn &

    At the end of the operations phase, an environ may have at most
    two stacks, one with characters eligible for missions and their
    spaceships and another for the rest of the player's forces
    [9.5]. &

    Attempt to end operations phase with three stacks in an environ &

    ??? Combine forces automatically or prompt user to refactor
    forces??? [16.14]

    \\ \hline 

    \rn &

    Character's stacked with military units must have leadership
    rating of at least 1 to be named leader [9.52]. &

    Attempt to name leader with less than 1 leadership rating. &
    
    Character does not assume leadership of stack.

    \\ \hline

    \rn &

    If a stack has no military units and multiple spaceships, the each
    spaceship must move as a separate stack and undergo any detection
    routines accordingly [9.54]. &

    Attempt to move a stack with two spaceships and no military
    units. &
    
    Movement is not allowed until a spaceship is specified. 

    \\ \hline 

    \rn &
    
    A spaceship may not be moved unless it is stacked with military
    units or a character with navigation rating of 1 or higher
    [9.55]. &

    Attempt to move a spaceship stacked with only a character with
    navigation rating less than 1 &

    Spaceship does not move.

 
    \\ \hline

    \rn &

    If at the end of the operations phase, a character is not stacked
    in the mission stack, character cannot perform missions during the
    missions phase [9.57]. &

    During missions phase, attempt to add character not stacked in the
    mission stack to a mission group. &
    
    Character is not added to mission group.

    \\ \hline 

  \end{longtable}

\end{center}

\newpage
\subsubsection{Turn Integration}

\setcounter{rc}{0}

\begin{center}

  \begin{longtable}{| p{.5cm} | p{4.5cm} | p{4.5cm} | p{4.5cm} |}
    \hline
    \textbf{\#}&
    \textbf{Rule Description}&
    \textbf{Test Description}&
    \textbf{Expected Result}

    \\ \hline
    
    \rn &
    
    If imperial player has less spaceships at the end of their turn
    than at the beginning of the game, \textit{and} they have an
    imperial knight on an imperial controlled planet, they can receive
    a spaceship [14.61 \& 15.73]. &
    
    End imperial turn with an imperial knight on an imperial
    controlled planet and with less spaceships in play than at the
    beginning of the game. &

    Imperial player has the option to receive a spaceship. 

    \\ \hline

  \end{longtable}

\end{center}


\newpage
=======
%\subsection{Capture}

\setcounter{rc}{0}

\begin{center}

  \begin{longtable}{| p{.5cm} | p{4.5cm} | p{4.5cm} | p{4.5cm} |}
    \hline
    \textbf{\#}&
    \textbf{Rule Description}&
    \textbf{Test Description}&
    \textbf{Expected Result}
    \\ \hline
    
    \rn &

    A unit must be assigned to "guard" a captured opponent [12.81]. &

    Succeed in capure combat. &
    
    Captured character is assgned a guard, either automatically or by
    prompthing user.

    \\ \hline 
    \rn &

    Captured character must be moved with the character assigned to
    "guard" them [12.81]. &

    Move character "guarding" a captured character. &

    Captured character moves to location guard moves to.
    
    \\ \hline

    \rn &

    Guarding character cannot perform missions [12.81]. &

    Attempt to assign guard to mission group. &

    Guard does not occupy mission group.

    \\ \hline 

    \rn &
    
    If a character is captured and combat is still being resolved, the
    captured character contributes nothing to either side [12.84]. &

    Successfully capture during capture combat. &

    Combat differential is refactored to remove captured character's
    benefits to forces.

    \\ \hline

  \end{longtable}

\end{center}

%\newpage
%\subsubsection{Combat}

\setcounter{rc}{0}

\begin{center}

  \begin{longtable}{| p{.5cm} | p{4.5cm} | p{4.5cm} | p{4.5cm} |}
    \hline
    \textbf{\#}&
    \textbf{Rule Description}&
    \textbf{Test Description}&
    \textbf{Expected Result}
    \\ \hline

    \rn &

    If a character stack in an environ is detected and there are no
    friendly military forces present, if they successfully break off
    from combat, they are no longer found and are no longer a part of
    combat [12.51]. &

    Break off from combat. &

    Character state is no longer found and combat is terminated.

    \\ \hline

    \rn & 

    If a character receives cumulative damage equal to their number of
    endurance points that character is dead and is removed from play
    [12.73]. &

    Engage in combat that results in enough wounds to match a
    character's endurance and assign those wounds to that character. &
    
    Character is removed from play.

    \\ \hline

    \rn &
    
    If attacking group is a squad, combat is firefight [12.31]. &

    Attack with squad and check combat type.&
    
    Combat type is firefight. 

    \\ \hline
    
    \rn & 
    
    Inactive characters do not contribute to combat rating or suffer
    damage [12.42]. &

    Engage in combat and set one character to inactive. &
    
    Character's stats do not contribute to combat rating and cannot
    have wounds assigned to them after battle.

    \\ \hline

    \rn & 
    
    Attacking character does not have inactive forces [12.44]. &

    Attack enemy forces. &

    No option to set inactive forces.

    \\ \hline

    \rn & 
    
    Leaders are not eliminated in military combat but if their forces
    are destroyed they are attacked by a squad [10.45]. &

    Engage in military combat with a leader where all forces are
    destroyed. &

    Leader is not destroyed but is attacked by a squad. 

    \\ \hline
    
    \rn &
    
    Characters stacked with military units are not effected by
    military combat and do not effect military combat but if their
    military units are eliminated, they are attacked by a squad
    [10.6]. &

    Engage in military combat with a stack containing characters where
    all military forces are eliminated. &

    Characters do not effect military combat and are not eliminated
    but are engaged in squad combat.

    \\ \hline

  \end{longtable}

\end{center}

%\newpage
%\subsubsection{Game Start}
\begin{center}

  \begin{longtable}{| p{.5cm} | p{4.5cm} | p{4.5cm} | p{4.5cm} |}
    \hline
    \textbf{\#}&
    \textbf{Rule Description}&
    \textbf{Test Description}&
    \textbf{Expected Result}
    \\ \hline
    
    1 &

    At game start, if player controls a planet, they may distribute
    characters and military forces on environs of those planets
    [16.12] &

    Player attempts to place a unit on the environ of a planet the
    player does not control &

    Unit does not occupy specified environ.  ??? Fail silently or
    notify player ???

    \\ \hline

    2 &

    At game start, if player does not control a planet, characters
    will "arrive from space" at the beginning of their operations
    phase. [16.12] &

    Player attempts to place units on environ of a planet they do not
    control &

    Unit does not occupy specified environ.  ??? Fail silently or
    notify player ???

    \\ \hline 

    3 &

    At game start, Imperial player places units first [16.14] &

    Rebel player attempts to place units \textit{before} Imperial
    player &

    Rebel player is unable to place forces.

    \\ \hline

    4 &

    At game start, Imperial player places units first [16.14] &

    Rebel player attempts to place units \textit{while} Imperial
    player \ is distributing their forces &

    Rebel player is unable to place forces.
    
    \\ \hline

    5 &
    Rebel game turn is first [16.14] &

    At start of game Imperial player attempts to begin turn
    \textit{before} Rebel player had begun turn &
    
    Imperial player is unable to begin turn.

    \\ \hline 

    6 &
    Rebel game turn is first [16.14] &

    At start of game Imperial player attempts to begin turn
    \textit{while} Rebel player is in their turn &

    Imperial player is unable to begin turn.

    \\ \hline 

    7 &

    At start of game, all friendly units on the same environ are
    placed in the same stack [16.13]. &

    Try and create a second stack in an environ at the beginning of
    play &

    Second stack is not constructed.

    \\ \hline

    8 &

    At start of game, if rebel player has units to distribute, units
    placed must match environ type [16.13]. &

    Attempt to place rebel units in environ that does not match their
    type. &

    User unable to place forces.

    \\ \hline 

  \end{longtable}

\end{center}

%\newpage
%\subsubsection{Missions}

\setcounter{rc}{0}

\begin{center}

  \begin{longtable}{| p{.5cm} | p{4.5cm} | p{4.5cm} | p{4.5cm} |}
    \hline
    \textbf{\#}&
    \textbf{Rule Description}&
    \textbf{Test Description}&
    \textbf{Expected Result}
    \\ \hline
    
    \rn &

    Diplomacy missions by a rebel player on a patriotic planet have 2
    bonus draws subtracted [13.2]. &

    Undergo diplomacy mission on a patriotic planet. &
    
    Total bonus draws are reduced by two. 

    \\ \hline 
    
    \rn &

    Diplomacy missions by a rebel player on a planet in dissent have 1
    bonus draw added [13.2]. &

    Undergo diplomacy mission on a planet in descent. &
    
    Total bonus draws are increased by one. 

    \\ \hline

    \rn &

    Subvert troops, start rebel camp and scavenge missions can only be
    performed by the rebel player [13.2]. &

    During the missions phase as Imperial player, view available
    missions. &

    Scavenge, start rebel camp and subvert troops missions are not
    available options.

    \\ \hline 
    \rn &

    Diplomacy missions by a rebel player on a patriotic planet have 2
    bonus draws subtracted [13.2]. &

    Undergo diplomacy mission on a patriotic planet. &
    
    Total bonus draws are reduced by two. 

    \\ \hline 

    \rn &
    
    A rebel camp is equivalent to a mission group for mission purposes
    [13.2]. &

    Assign rebel camp a mission in an environ during the mission
    phase. &

    Rebel is assigned mission. 

    \\ \hline

    \rn &
    
    Rebel camps cannot receive bonus draws [13.2]. &

    Draw action card resulting in a bonus draw with only a rebel camp
    doing missions in an environ. &

    No bonus draws are awarded. 

    \\ \hline 

    \rn &
    
    Rebel camps are never effected by the effects of an action card. &

    Perform mission with rebel camp and draw an action card resulting
    in combat. &

    Rebel camp does not engage in combat. 
 
    \\ \hline

    \rn &

    There can never be more than one rebel camp in an environ [13.2]. &

    Attempt start rebel camp mission in an environ with a rebel camp. &

    Mission is not available. 

    \\ \hline 

    \rn &
    
    Rebel camps cannot be moved [13.2]. &

    View movable units in an environ with a rebel camp during
    operations phase. &

    Rebel camp is not displayed. 

    \\ \hline 

    \rn &
    
    Characters cannot accompany rebel camps on a mission [13.2]. &

    Break mission stack into mission groups and attempt to add a
    character to a rebel camp mission group. &
    
    Character does not occupy rebel camp's mission group. 
    
    \\ \hline 

    \rn &

    Spaceship quest, summon sovereign, question prisoner and steal
    resources missions are not available in the star system games
    [13.2]. &

    View available missions during missions phase of a star system
    game.&
    
    Spaceship quest, summon sovereign, question prisoner and steal
    resources missions are not available.
    
    \\ \hline 

    \rn &

    Maximum number of action cards drawn for mission groups in an
    environ is equal to the environ size [13.3]. &

    Draw four action cards in an environ of size four. &

    Only bonus draws remain.

    \\ \hline  

    \rn &

    Bonus draws are only available is mission group has drawn maximum
    number of action cards for the environ size [13.3].&

    Quit missions in an environ of size 4 after drawing 1 action card
    when a character has bonus draws for the mission type. &

    No bonus draws available. 
    
    \\ \hline

    \rn &

    Action card must be resolved before mission letter can take effect
    [13.3]. &

    Die in combat with creatures resulting from an action card that
    would have otherwise resulted in mission success. &
    
    Mission does not succeed. 
    
    \\ \hline

    \rn &

    Except for the case of the NPP searching for characters on a mission, if
    an action drawn is the same as an action previously drawn in that
    environ, the action is ignored [13.42]. &

    Draw an action for creature combat after creature combat has
    already been resolved during that turn in that environ. &

    Mission group does not engage in creature combat. 

    \\ \hline

    \rn &

    In the case of an action card that allows the NPP to search for
    characters on a mission, this action may be repeated \textit{n}
    times where \textit{n} is the number of mission groups in that
    environ [13.48].  &

    Draw action card for two mission groups resulting in NPP search
    after NPP has already searched once in the environ.  &

    NPP performs search for mission groups in current mission environ.

    \\ \hline

    \rn &

    In the case of an action card that allows the NPP to search for
    characters on a mission, this action may be repeated \textit{n}
    times where \textit{n} is the number of mission groups in that
    environ [13.48].  &

    Draw action card for two mission groups resulting in NPP search
    after NPP has already searched twice in the environ. &

    NPP does not perform a search.    
    
    \\ \hline

    \rn &

    If an action card is drawn that contradicts an action drawn
    previously for a mission group in an environ in a turn, the
    contradictory action is ignored [13.42]. &

    Draw an action card that prohibits bonus draws after drawing an
    action card that grants one extra bouns draw. &

    Extra bonus draw is kept, bonus draws are not prohibited. 

    \\ \hline
    
    \rn &

    If a single mission group is affected by an action card, that
    mission group is randomly chosen from all active mission groups in
    that environ [13.43]. &

    Draw action card resulting in creature combat in an environ
    containing two mission groups. &

    One of the two mission groups is chosen at random to engage in
    creature combat.
    
    \\ \hline

    \rn &

    If no creature is named by an environ the PP is attacked by sentry
    robots only if the NPP controls the planet [13.46]. &

    Draw creature attack in an environ where no creature is named and
    the NPP controls the planet. &

    Player engages in combat with sentry robots. 

    \\ \hline

    \rn &

    If no creature is named by an environ the PP is attacked by sentry
    robots only if the NPP controls the planet [13.46]. &

    Draw creature attack in an environ where no creature is named and
    the NPP does not control the planet. &

    Player does not engage in combat.
    
    \\ \hline
    
    \rn &

    If a planet is in a state of rebellion or in rebel control, the
    rebel player ignores irate locals attacks [13.47]. &

    As the rebel player performing a mission on a planet under rebel
    control, draw action card resulting in irate locals attack. &
    
    Player does not engage in combat. 
    
    \\ \hline

    \rn &

    If a planet is in a state of rebellion or in rebel control, the
    rebel player ignores irate locals attacks [13.47]. &

    As the rebel player performing a mission on a planet in rebellion,
    draw action card resulting in irate locals attack. &
    
    Player does not engage in combat. 
    
    \\ \hline
    
    \rn &

    If an action card is drawn that results in characters being found
    and there are NPP forces in that mission environ, the NPP's forces
    must attack one mission group. [12.15] &
    
    Draw an action card resulting in characters found in an environ
    where there are enemy fores. &

    Combat is initiated by either enemy characters or enemy squad at
    NPPs choice.

    \\ \hline
    
    \rn &

    If an action card is drawn that results in characters being found
    and there are no NPP forces in that mission environ, characters
    are not detected. [12.15] &
    
    Draw an action card resulting in characters found in an environ
    where there are no enemy fores. &

    Characters are not found. 

    \\ \hline

    \rn &

    Gather information mission cannot be performed on a planet under
    control of PP is there are NPP characters or military units in the
    mission environ [15.52]. &

    View available missions on a planet under PP control in an environ
    where NPP military or detected character forces are present &

    No gather information mission available. 

    \\ \hline

    \rn &

    Coup and diplomacy missions cannot be performed on a planet in a
    state of rebellion or under rebel control [15.57]. &

    View available missions on a planet in rebellion or under rebel
    control &

    No coup or diplomacy missions available. 

    \\ \hline

    \rn &

    If a planet's state is "rebellion stopped," the loyalty marker and
    the rebel-control marker are moved together [15.75]. &

    Complete mission that results in shifting loyalty marker. & 

    Rebel-control marker shifts with loyalty marker. 

    \\ \hline

    \rn &
    
    A coup mission requires a character with an intelligence rating of
    at least 1 [13.2]. &

    Attempt to assign a coup mission to a mission group without a
    character with an intelligence rating of at least 1. &

    Mission is not assigned to mission group. 

    \\ \hline

  \end{longtable}

\end{center}

%\newpage
%\subsubsection{Movement}

\setcounter{rc}{0}

\begin{center}

  \begin{longtable}{| p{.5cm} | p{4.5cm} | p{4.5cm} | p{4.5cm} |}
    \hline
    \textbf{\#}&
    \textbf{Rule Description}&
    \textbf{Test Description}&
    \textbf{Expected Result}
    \\ \hline
    
    \rn &

    Characters can only move from planet to planet using a spaceship
    [9.0]. &

    Attempt to move characters without a spaceship stacked with mobile
    military units. &

    Characters are left on the environ.

    \\ \hline 
    \rn &

    Every unit/character can be moved at most once during operations
    phase [9.0]. &

    Attempt to move a unit/character after moving it once. &

    Unit/character does not move.
    
    \\ \hline

    \rn &

    Non-phasing player may make a reactin move only once the phasing
    player is finished with their operations phase [9.0]. &

    Attempt to move NPP units before PP is done with operations
    phase. &

    NPP units do not move. 

    \\ \hline 

    \rn &

    Non-phasing player may make a reactin move only once the phasing
    player is finished with their operations phase [9.0]. &

    Attempt to move NPP units during PP's operations phase. &

    NPP units do not move.     
    
    \\ \hline

    \rn &
    
    Military units cannnot move to an environ that is populated by a
    number of military units equal to it's size [9.5]. &

    Attempt to move a stack of one militart unit to an environ
    populated by a max number of military units.&

    Military unit does not occupy environ.

    \\ \hline 

    \rn &
    
    A spaceship cannot carry more characters than it's capacity stat
    dictates [9.56]. &

    Construct a character stack with a spaceship that contains more
    characters than the capacity of the spaceship dictates. Attempt to
    move stack to another environ. &

    Only the number of characters equaling the capacity of the
    spaceship move.
    
    \\ \hline

    \rn &
    
    The non-phasing player may make a move for each evviron containing
    PP military and detected forces [9.6]. &

    Enter reaction move phase when there are \textit{n} environs
    containing military units and detected forces. &

    Non-phasing player is allowed \textit{n} moves. 

    \\ \hline

    \rn &
    
    A reaction move cannot be made from one planet to another
    [9.61]. &

    Attempt to move a stack from planet to planet for a reaction
    move. &
    
    Stack does not occupy new environ.
    
    \\ \hline
    
    \rn &

    &

    &

    


    \\ \hline

  \end{longtable}

\end{center}

%\newpage
%\subsubsection{PDBs and Detection}

\setcounter{rc}{0}

\begin{center}

  \begin{longtable}{| p{.5cm} | p{4.5cm} | p{4.5cm} | p{4.5cm} |}
    \hline
    \textbf{\#}&
    \textbf{Rule Description}&
    \textbf{Test Description}&
    \textbf{Expected Result}
    \\ \hline
    
    \rn &

    Every planet has a PDB [8.0]. &

    Initialize galaxy. &
    
    Every planet has a PDB. 

    \\ \hline 
    \rn &

    The level of a PDB is  0, 1 or 2 [8.0]. &

    Initialize galaxy. &

    PDB levels conform to states 0, 1 or 2
    
    \\ \hline

    \rn &

    PDBs are either up or down [8.0]. &

    Initialize Galaxy. &

    PDBs conform to states up or down.

    \\ \hline 

    \rn &
    
    In Star System Games, the level of a PDB cannot be improved
    [8.13]. &

    ( ???? ) Succeed in a mission ( ???? ) that would improve PDB
    level. &
    
    PDB level does not improve.
    

    \\ \hline

    \rn &
    
    PDB can only be used if its status is "up" [8.2]. &

    Move characters from an environ on a planet where the PDB is down
    in a spaceship. &

    Characters do not undergo detection routine. 

    \\ \hline 

    \rn &
    
    If characters move across a planet without a spaceship they do not
    undergo the detection routine [9.0]. &

    Move character stack from one environ to another on the same
    planet. &
    
    Characters do not undergo detection routine. 
 
    \\ \hline

    \rn &

    If characters move across a planet with a spaceship may undergo a
    detection routine [9.0]. &

    Move character stack from one environ to another on a planet with
    a PDB up and at level 1 or higher .&

    Character stack undergoes detection routine. 

    \\ \hline 

    \rn &
    
    If a PDB incurs a loss of 2 when attacking it is placed in the
    "down" state [9.25]. &

    Incur a loss of 2 with an attacking PDB. &

    PDB state is "down."
 
    \\ \hline

    \rn &

    If a PDB incurs a loss of 3 when attacking it is placed in the
    "down" state \textit{and} it is reduced a level [9.25]. &

    Incur a loss of 3 with an attacking PDB. &

    PDB state is "down" \textit{and} level is decremented by 1.

    \\ \hline 

    \rn &
    
    A loss of 1 incurred by a PDB does nothing [9.25]. &

    Incur a loss of 1 with an attacking PDB. &
    
    PDB does not change state. 
 
    \\ \hline

    \rn &

    If a spacehip leaves a planet unable to conduct a detection
    routine, it and its characters are no longer detected [9.45]. &

    Move detected character stack in a spaceship from one environ to
    another on a planet unable to conduct a detection routine. &

    Characters are no longer detected. 

    \\ \hline 

    \rn &

    Stacks with military units are detected. &

    Move military units into a stack of undetected characters in an
    environ. &

    Characters are detected. 

    \\ \hline 

    \rn &

    If a detected character stack leaves an environ "undetected" by
    the enemy PDB they are no longer detected [9.31]. &
    
    Move detected character stack in a spaceship from an envrion on a
    planet with an active PDB where the detection routine results in
    "undetected". &

    Characters are no longer detected. 
    Character stack is no longer detected. 

    \\ \hline 

    \rn &

    A PDB with a level 0 only detects regardless of the outcome of the
    detection routine [9.32]. &

    Undergo detection routine with a PDB at level 0 resulting in
    "eliminated." &

    Character stack is detected.

    \\ \hline 

    \rn & 

    Detected characters moving from one environ to another
    containing no detected characters or military units on the same
    planet without a spaceship are no longer detected [9.35]. & 

    Move detected characters to another environ on the same planet
    that contains no detected characters and no military units. &
    
    Characters are no longer detected. 

    \\ \hline 

    \rn &

    Military units are always detected but can be attacked by a PDB. &

    Move military units from space into a planet's environ. &

    Military units undergo detection routine and are subject to attack
    accoring to the military combat results table.
    
    \\ \hline 

    \rn &

    If a spaceship is "detected and damaged" twice in the same turn,
    the second "detected and damaged" outcome is equivalent to
    "eliminated" outcome [9.35]. &

    Move from one environ to another on a planet with an active PDB at
    level 1 or more and incur "detected and damaged" outcome twice. &

    Characters and spaceship are eliminated. 

    \\ \hline 

    \rn &

    In the star system games, if a detection routine results in
    "detected and damaged," the spaceship is eliminated when the
    destination is reached [9.31]. &

    Undergo detection routine that results in "detected and damaged" &
    
    Spaceship is eliminated on arrival. 

    \\ \hline 

    \rn &
    
    If a detection routine results in "eliminated," the spaceship and
    all characters on board are eliminated immediately [9.31]. &
    
    Undergo detection routine that results in "eliminated" &

    Spaceship and characters in it are eliminated. 
    
    \\ \hline 

  \end{longtable}

\end{center}

%\newpage
%\subsubsection{Planets and Control}

\setcounter{rc}{0}

\begin{center}

  \begin{longtable}{| p{.5cm} | p{4.5cm} | p{4.5cm} | p{4.5cm} |}
    \hline
    \textbf{\#}&
    \textbf{Rule Description}&
    \textbf{Test Description}&
    \textbf{Expected Result}
    \\ \hline
     
    \rn &

    At the beginning of play, every planet has a loyalty counter in 1
    of five states [15.1] &

    Check every planet's loyalty after game starts. &
    
    Loyalty counters are not null.  

    \\ \hline 
    \rn &

    A loyalty counter at patriotic cannot be increased further in the
    Imperial player's favor [15.13]. &
    
    Successful Imperial "Diplomacy" or "Political Tract" mission when
    loyalty counter is set to patriotic. &

    Loyalty counter does not change state. 
    
    \\ \hline

    \rn &

    A loyalty counter at unrest cannot be increased further in the
    Rebel player's favor [15.13]. &
    
    Successful Rebel "Diplomacy" or "Political Tract" mission when
    loyalty counter is set to patriotic. &

    Loyalty counter does not change state. 
   
    \\ \hline

    \rn &
    
    Loyalty counter on a planet in a state of rebellion or rebel
    control cannot be moved [15.14]. &

    Successful Rebel "Diplomacy" or "Political Tract" mission when
    planet is in a state of rebellion or rebel control. &

    Loyalty counter does not change state. 

    \\ \hline

    \rn &
    
    A planet in a state of rebellion is controlled by neither player
    [15.4]. &

    Put a planet in the state of rebellion. &

    "Control indicator" is in neither a rebel or imperial state.

    \\ \hline 

  \end{longtable}

\end{center}

%\newpage
%\subsubsection{Searching}

\setcounter{rc}{0}

\begin{center}

  \begin{longtable}{| p{.5cm} | p{4.5cm} | p{4.5cm} | p{4.5cm} |}
    \hline
    \textbf{\#}&
    \textbf{Rule Description}&
    \textbf{Test Description}&
    \textbf{Expected Result}
    \\ \hline
    
    \rn &
    
    Search is conducted with either military units or
    characters. Characters stacked with military units cannot be used
    to conduct character search unless they are moved from the
    military stack [11.21]. &

    Attempt to perform character search with character stacked with
    military. &

    ??? Either search is not conducted or a military search is
    conducted instead. ???

    \\ \hline 

    \rn &

    Only the successful search group attacks [11.23].&

    Successfully search with characters. & 

    Only character combat should be initiated.
    
    \\ \hline

    \rn &

    Using characters to search results in their detection [11.31]. &

    Search with undetected characters. &11

    Characters become detected. 

    \\ \hline 

    \rn &
    
    Searching may only be done in environs where a player has units
    and there are \textit{detected} enemy forces [11.1].&

    Attempt to search an environ where there are no detected forces. &

    No search is conducted. 

    \\ \hline 
  \end{longtable}

\end{center}

%\newpage
%\subsubsection{Stacking}

\setcounter{rc}{0}

\begin{center}

  \begin{longtable}{| p{.5cm} | p{4.5cm} | p{4.5cm} | p{4.5cm} |}
    \hline
    \textbf{\#}&
    \textbf{Rule Description}&
    \textbf{Test Description}&
    \textbf{Expected Result}
    \\ \hline
    
    \rn &

    Number of military units in a stack, in an environ can never
    exceed the environ size [9.5] &

    Try and place a military unit in an environ that is at capacity &
    
    Military unit is not added to the stack.

    \\ \hline 
    \rn &

    There is no limit to the number of units in a stack [9.5]. &
    
    Construct an artificial stack out of all military units + all
    characters + all ships&
    
    All units, characters, ships occupy the same stack.
    
    \\ \hline

    \rn &

    At the end of the operations phase, an environ may have at most
    two stacks, one with characters eligible for missions and their
    spaceships and another for the rest of the player's forces
    [9.5]. &

    Attempt to end operations phase with three stacks in an environ &

    ??? Combine forces automatically or prompt user to refactor
    forces??? [16.14]

    \\ \hline 

    \rn &

    Character's stacked with military units must have leadership
    rating of at least 1 to be named leader [9.52]. &

    Attempt to name leader with less than 1 leadership rating. &
    
    Character does not assume leadership of stack.

    \\ \hline

    \rn &

    If a stack has no military units and multiple spaceships, the each
    spaceship must move as a separate stack and undergo any detection
    routines accordingly [9.54]. &

    Attempt to move a stack with two spaceships and no military
    units. &
    
    Movement is not allowed until a spaceship is specified. 

    \\ \hline 

    \rn &
    
    A spaceship may not be moved unless it is stacked with military
    units or a character with navigation rating of 1 or higher
    [9.55]. &

    Attempt to move a spaceship stacked with only a character with
    navigation rating less than 1 &

    Spaceship does not move.

 
    \\ \hline

    \rn &

    If at the end of the operations phase, a character is not stacked
    in the mission stack, character cannot perform missions during the
    missions phase [9.57]. &

    During missions phase, attempt to add character not stacked in the
    mission stack to a mission group. &
    
    Character is not added to mission group.

    \\ \hline 

  \end{longtable}

\end{center}

%\newpage
%\subsubsection{Turn Integration}

\setcounter{rc}{0}

\begin{center}

  \begin{longtable}{| p{.5cm} | p{4.5cm} | p{4.5cm} | p{4.5cm} |}
    \hline
    \textbf{\#}&
    \textbf{Rule Description}&
    \textbf{Test Description}&
    \textbf{Expected Result}

    \\ \hline
    
    \rn &
    
    If imperial player has less spaceships at the end of their turn
    than at the beginning of the game, \textit{and} they have an
    imperial knight on an imperial controlled planet, they can receive
    a spaceship [14.61 \& 15.73]. &
    
    End imperial turn with an imperial knight on an imperial
    controlled planet and with less spaceships in play than at the
    beginning of the game. &

    Imperial player has the option to receive a spaceship. 

    \\ \hline

  \end{longtable}

\end{center}


%\newpage

>>>>>>> f48c05a3a81d528f5fafa9566de8530f216836cd

\subsection{System Testing Pass/Fail Criteria}
TBD
%\input{}

\subsection{User Interface Manual Testing Pass/Fail Criteria}
This is up to the client team?
%\input{}

\section[SUSPENSION CRITERIA]{SUSPENSION CRITERIA}
The automated testing procedure shall be completed in the following order:
\begin{enumerate}
\item Unit Testing
\item Integration Testing
\item System Testing
\end{enumerate}
Each item to be tested is required to pass each unittest of a classification of 1 with 100\% success before it can be included in a higher level of testing. This complete success can be waived for the following reasons:
\begin{itemize}
\item Feature under test will not be included in the upcoming deliverable.
\item Feature under test will not be included in the final product.
\item Feature under test is complex and meeting the test requirements will delay the deliverable.
\end{itemize}
Decision to allow a feature to proceed to a higher level of testing shall be determined by team leader.

\section[TEST DELIVERABLES]{TEST DELIVERABLES}
\begin{itemize}
\item Test Cases
\item Test Logs
\item Incident Reports
\item Outputs
\item Corrective Actions
\end{itemize}

\subsection{Test Cases}

All unittests, integration tests, and system tests should be documented in TCSs. The name of a TCS should match the module name (e.g. Character Class TCS) or functions (e.g. Main Menu TCS). TCS should be developed from any requirements document and should tests every requirement of those documents.

\section[REMAINING TEST TASKS]{REMAINING TEST TASKS}
<<<<<<< HEAD
None
=======
\begin{itemize}
\item Divsion of Labor by each group leader.
\item Implementation of testing.
\item Complication of testing results.
\end{itemize}
>>>>>>> f48c05a3a81d528f5fafa9566de8530f216836cd

\section[EVIRONMENTAL NEEDS]{ENVIRONMENTAL NEEDS}
None

\section[STAFFING AND TRAINING NEEDS]{STAFFING AND TRAINING NEEDS}
Training on portions of the project shall be carried out by the authors of the code and the documented design and the responsibility of said authors. Help is available from other members of the team.


\section[RESPONSIBILITIES]{RESPONSIBILITIES}
<<<<<<< HEAD
The responsibility of this document is not yet determined?
=======
The responsibility of this document is the entire Python Group
>>>>>>> f48c05a3a81d528f5fafa9566de8530f216836cd


\section[SCHEDULE]{SCHEDULE}
\begin{center}
\begin{tabularx}{\textwidth}{| X | X | X |}
  \hline
  \textbf{Deliverable} &
  \textbf{Description} &
  \textbf{Due Date} 
\\ \hline
Test Plan version 1.0 & 
Completion of first draft of the complete Test Plan documentation. &
11/12/2013
\\ \hline

Test Assigment &
Assign tests to team members &
11/14/2013
\\ \hline

Delivery of Unit Tests &
Completion of assigned Unit Tests &
11/25/2013
\\ \hline

Delivery of Integration Tests &
Completion of assigned Integration Tests &
12/5/2013
\\ \hline

Delivery of System Tests and Manual Tests &
Completion of assigned System Tests &
12/10/2013
\\ \hline

\end{tabularx}
\end{center}

\section[PLANNING RISKS AND CONTINGENCIES]{PLANNING RISKS AND CONTINENCIES}
Because the end of the semester is fixed, there are no contingencies if the product does not meet requirements by that date. Public beatings will be carried out as needed.

\section[APPROVALS]{APPROVALS}
<<<<<<< HEAD
Approve is up to team leaders.
=======
Approval is everyones responsibility since all students will be evaluated by this document.
>>>>>>> f48c05a3a81d528f5fafa9566de8530f216836cd

\section[GLOSSARY]{GLOSSARY}
SCR - Software Change Request
TCS - Test Specifications Document

\end{document}
