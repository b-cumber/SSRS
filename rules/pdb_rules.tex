\section{PDB}
\label{sub:pdb}

% status : started

\subsection{PDB}
For rules regarding detection, see section \ref{sub:detection}. For rules regarding PDB combat, see section \ref{sub:pdbcombat}.

\setcounter{rc}{0}

\begin{center}

  \begin{longtable}{| p{\first} | p{\second} | p{\third} | p{\fourth} |}
    \hline
    \textbf{\#}&
    \textbf{Rule Description}&
    \textbf{Reference(s)}&
    \textbf{Metric}
    \\ \hline
    
    \newrule{The PDB is controlled by the player who controls the planet.}{8.0}{7}
    
    \newrule{Every planet has a PDB.}{8.0}{3}
    
    \newrule{The level of the PDB is one of the following values: 0,1,2.}{8.0}{2}
    
    \newrule{The PDB can have status of �up� or �down�.}{8.0}{2}
    
    \newrule{A status of �up� means functional and a status of �down� means not functioning.}{8.0}{1}
    
    \newrule{If the planet is not in control of either player, the PDB cannot be used by any player.}{8.33}{3}
    
    \newrule{The level of a PDB cannot be improved.}{8.13}{1}
    
    \newrule{A PDB can be used to attempt detection of Enemy characters entering, leaving, or traveling on the planet in a spaceship.}{8.2}{8}
    
    \newrule{The PDB can only be used if the status is �up�.}{8.2}{4}
    
    \newrule{A PDB with a status of �down� cannot be used for any purpose.}{8.2}{3}
    
    \newrule{The PDB can change to �down� status due to �Sabotage Mission� (n53), Action Event �Locals Raid� (n70).}{8.31}{6}
    
    \newrule{If units move from environ to another environ on the same planet, the PDB cannot be used for detection.}{9.0}{3}
    
    \newrule{If characters move across the planet without a spaceship, the PDB cannot be used for detection.}{9.0}{2}
    
    \newrule{If the characters move across the planet with a spaceship, the characters are considered to be in the spaceship and can be detected.}{9.0}{4}
    
  \end{longtable}
\end{center}
