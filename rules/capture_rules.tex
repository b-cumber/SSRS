\section{Capture}
\label{sub:capture}

% status: completed

\setcounter{rc}{0}

\begin{center}

  \begin{longtable}{| p{\first} | p{\second} | p{\third} | p{\fourth} |}
    \hline
    \textbf{\#}&
    \textbf{Rule Description}&
    \textbf{Reference(s)}&
    \textbf{Metric}
    \\ \hline
    
    \rn &
    
    Intention must be of one of two types in combat: �kill� or �capture�. &
    
    12.3 &
    
    -
    
    \\ \hline
    
    \rn &
    
    There is a two column shift to the left for capture combat.  &
    
    12.5 &
    
    -
    
    \\ \hline
    
    \rn &
    
    If the attacker announces capture combat, the differential is shifted two to the left.  &
    
    12.6 &
    
    -
    
    \\ \hline 
    
    \rn &
    
   The decision to capture or kill applies to all rounds of combat. &
   
    12.8 &  
    
    -
    
    \\ \hline
    
    \rn &
    
    Capture occurs after all wounds have been assigned. &
    
    12.8 &
    
    -
    
    \\ \hline
    
    \rn &
    
    If a character is captured, the character is chosen randomly from all active characters. &
    
    12.8 &
    
    -
    
    \\ \hline
    
    \rn &
    
    A unit must be assigned to guard the captured character. &
    
    12.81 &
    
    -
    
    \\ \hline
    
    \rn &
    
    If the captured character is guarded by an enemy character, that guarding character cannot perform missions. &
    
    12.81 &
    
    -
    
    \\ \hline
    
    \rn &
    
    If the captured character is guarded by an enemy military unit, that military unit is unaffected. &
    
    12.81 &
    
    -
    
    \\ \hline
    
    \rn &
    
    The captured character never contributes anything to enemy units. &
    
    12.81 &
    
    -
    
    \\ \hline
    
    \rn &
    
    The captured character can be moved at any time during the movement phase to any other character or military unit in the same environ. &
    
    12.81 &
    
    -
    
    \\ \hline
    
    \rn &
    
    A captured character unit can be freed by executing the �Free Prisoners� mission. &
    
    12.82 &
    
    -
    
    \\ \hline
    
    \rn &
    
    A captured character unit can also be freed if the captured character is not stacked with any enemy units at any time. &
    
    12.82 &
    
    -
    
    \\ \hline
    
    \rn &
    
    If the capturing force is eliminated in the same combat, the character is not considered captured. &
    
    12.83 &
    
    -
    
    \\ \hline
    
    \rn &
    
    If a character is captured and combat is still being resolved, the captured character contributes nothing to either side. &
    
    12.84 &
    
    - 
    
    \\ \hline
    
    \rn &
    
    A player may only declare capture at the beginning of combat and if he is the attacker.  &
    
    12.85 &
    
    -
    
    \\ \hline
    
    \rn &
    
    The * on the combat results table are ignored during kill type combat. &
    
    12.86 &
    
    -
    
    \\ \hline
    
    \rn &
    
    If during the action event, �Coup Mission aborted�, or the failure to complete an �Assassination� mission, there are no enemy units in the same environ, the characters are not captured. &
    
    12.87 &
    
    -
    
    
    \\ \hline
  \end{longtable}
\end{center}
