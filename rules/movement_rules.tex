\section{Movement}

\setcounter{rc}{0}

\begin{center}

  \begin{longtable}{| p{\first} | p{\second} | p{\third} | p{\fourth} |}
    \hline
    \textbf{\#}&
    \textbf{Rule Description}&
    \textbf{Reference(s)}&
    \textbf{Metric}
    \\ \hline
    
    \newrule{During the movement segment of the player�s turn, a player can move any of his characters, character spaceships, and military units.}{9.0}{-}
    
    \newrule{Units can be moved from an Environ to a environ on the same planet without restriction.}{9.0}{-}
    
    \newrule{Characters can only move from an environ to an environ on a different planet by using a spaceship.}{9.0}{-}
    
    \newrule{Mobile units can move from an environ to an environ on a different planet without restriction.}{9.0}{-}
    
    \newrule{A unit can only move to another environ, i.e. cannot move a non?environ space such the orbit box or drift.}{9.0}{-}
    
    \newrule{If units are moved together, they are referred to as a stack.}{9.0}{-}
    
    \newrule{Moving units has no penalty due to distance or type of movement (e.g. interplanetary).}{9.0}{-}
    
    \newrule{Each unit can only be moved at most once during the movement phase.}{9.0}{-}
    
    \newrule{The non?phasing player can make a reaction move only when the phasing player has finished their movement segment.}{9.0}{-}
    
    \newrule{The player most choose the destination and continue to the destination unless destroyed in the process, i.e. any event after the decision to move cannot influence the destination.}{ }{-}
    
    \newrule{Military units cannot move to an environ that has reached its maximum environ size.}{9.5}{-}
  
  \end{longtable}
\end{center}