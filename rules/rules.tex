\documentclass[a4wide]{article}
\usepackage{a4wide}
\usepackage{tabulary}
\usepackage{longtable}
\usepackage{tabularx}
\usepackage{graphicx}
\usepackage
[
%        a4paper,% other options: a3paper, a5paper, etc
        left=2.7cm,
        right=2.7cm,
%        %top=3cm,
%        %bottom=4cm,
]
{geometry}

\newcommand{\comment}[1]{{\tt #1}}
\newcounter{rc}
\newcommand{\rn}{\stepcounter{rc}\therc}

\begin{document}

\title{\textsc{Rules Summary Document}
  \\ Freedom in the Galaxy, Java and Python Teams
  \vspace{10 mm}
  \\ Version 1.0
  \vspace{10 mm}}
  
\date{\today}

\author{Prepared for:
  \\ CS 383 Course Project
  \vspace{10 mm}
  \\Prepared by:
  \\ Documentation Subteams of Python and Java Teams 
  \vspace{10 mm}}
  
\maketitle
\newpage
  
\begin{center}
\noindent Freedom In The Galaxy Test Plan, Python Group

\vspace{10 mm}

\noindent \textsc{Record of Changes}   

\vspace{10 mm}


\begin{tabularx}{\textwidth}{| X | X | X | X | X | X |}
  \hline
  \textbf{Change Number} &
    \textbf{Date Completed} &
    \textbf{Location of Change (e.g. page \# or figure \#)} &
    \textbf{Brief Description of Change} &
    \textbf{Approved by (initials)} &
    \textbf{Date Approved} 
    \\ \hline 1 & Nov 24 & ALL  & INITIAL CONVERSION & - & -
%    \\ \hline & & & & &  
%    \\ \hline & & & & &  
%    \\ \hline & & & & &  
    \\ \hline
\end{tabularx}
\end{center}
\newpage 
\tableofcontents
\newpage

\section{Introduction}

This is the introduction. The following sections contain all the rules as provided by the Galactic Guide and Rules of Play, except for the last section \ref{sec:changes}, which contains all the modification to the stated rules. Metric refers to the points given for implementing the rule. Blah blah blah.


\section{Capture}

\setcounter{rc}{0}

\begin{center}

  \begin{longtable}{| p{.5cm} | p{10.0cm} | p{2.0cm} | p{2.5cm} |}
    \hline
    \textbf{\#}&
    \textbf{Rule Description}&
    \textbf{Reference(s)}&
    \textbf{Metric Value}
    \\ \hline
    
    \rn &
    
    Intention must be of one of two types in combat: �kill� and �capture�. &
    
    12.3 &
    
    -
    
    \\ \hline
    
    \rn &
    
    There is a two column shift to the left for capture combat.  &
    
    12.5 &
    
    -
    
    \\ \hline
    
    \rn &
    
    If the attacker announces capture combat, the differential is shifted two to the left.  &
    
    12.6 &
    
    -
    
    \\ \hline 
    
    \rn &
    
   The decision to capture or kill applies to all rounds of combat. &
   
    12.8 &  
    
    -
    
    \\ \hline
  \end{longtable}
\end{center}


\section{Control}

% status : completed, except for determining who controls the planet

\setcounter{rc}{0}

\begin{center}

  \begin{longtable}{| p{\first} | p{\second} | p{\third} | p{\fourth} |}
    \hline
    \textbf{\#}&
    \textbf{Rule Description}&
    \textbf{Reference(s)}&
    \textbf{Metric}
    \\ \hline
    
    \rn &
    
    Attack by Sentry robots only occurs if the non-phasing player controls the planet. &
    
    13.46, 15.51 &
    
    - 
    
    \\ \hline
    
    \rn &
    
    Gather information mission cannot be performed on a planet under control of the phasing player if there are enemy character or military units present. &
    
    15.52 &
    
    -
    
    \\ \hline
    
    \rn &
    
    If the Imperial player has less spaceships in play than at the beginning of the game and has any Imperial Knight on an Imperial?controlled planet, that Imperial player can received a spaceship, which is stacked with the Imperial Knight up to the beginning of the play limit. &
    
    14.61 &
    
    - 
    
    \\ \hline
    
    \rn &
    
    The number of spaceships received from rule 14.61 is limited by the number of starships obtained at the beginning of the play limit. &
    
    15.73 &
    
    -
    
    \\ \hline
    
    \rn &
    
    Characters do not have any effect and are ignored in determining who controls the planet. &
    
    15.74 &
    
    - 
    
    \\ \hline
    
    \rn &
    
    Neither player may perform �Coup� or �Diplomacy� missions if the planet is in a state of Rebel controlled. &
    
    15.75 &
    
    - 
    
    \\ \hline
    
    \rn &
    
    If a planet�s loyalty is shifted when a planet is in a state of Rebellion Stopped, the Loyalty marker and Rebel?control marker are moved together.  & 
    
    15.75 &
    
    - 
    
    \\ \hline
    
    \rn &
    
    A planet must be in one of four states: "Imperial Control", "Rebellion", "Rebel Control", "Rebellion Stopped". &
    
    15.5 &
    
    -
    
    \\ \hline
    
    \rn &
    
    Actual control of a planet is determined by military units and status of PDB. &
    
    15.5 &
    
    -
    
    \\ \hline
    
    \rn &
    
    Control of a planet may change at any time. &
    
    15.5 &
    
    -
    
        
    \\ \hline
  \end{longtable}
\end{center}

\section{Changes to the Official Rules}
\label{sec:changes}

References refers to the section of the rules modified. Reasons for these modifications are presented in the SSRS document.

\setcounter{rc}{0}

\begin{center}

  \begin{longtable}{| p{.5cm} | p{10.0cm} | p{2.0cm} | p{2.5cm} |}
    \hline
    \textbf{\#}&
    \textbf{Rule Description}&
    \textbf{Reference(s)}&
    \textbf{Metric Value}
    \\ \hline
    
    \rn &
    
    PDBs automatically roll a 20 against all character spaceships &
    
    9.3 &
    
    999999
    
    \\ \hline
  \end{longtable}
\end{center}


%Intention must be of one of two types in combat: �kill� and �capture�. [12.3]  \\
%
%There is a two column shift to the left for capture combat. [12.5]  \\
%
%If the attacker announces capture combat, the differential is shifted two to the left. [12.6] \\
%
%The decision to capture or kill applies to all rounds of combat [12.8] \\
%
%The capture occurs after all wounds have been assigned. [12.8]
%
%If a character is captured, the character is chosen randomly from all active characters. [12.8]
%
%A unit must be assigned to guard the captured character. [12.81]
%
%The captured character must be moved with a unit. [12.81]
%
%If the captured character is guarded by an enemy character, that guarding character cannot perform missions. [12.81]
%
%If the captured character is guarded by an enemy military unit, that military unit is unaffected. [12.81] 
%
%The captured character never contributes anything to enemy units. [12.81]
%
%The captured character can be moved at any time during the movement phase to any other character or military unit in the same environ. [12.81]
%
%A captured character unit can be freed by executing the �Free Prisoners� mission. [12.82]
%
%A captured character unit can also be freed if the captured character is not stacked with any enemy units at any time. [12.82]
%
%If the capturing force is eliminated in the same combat, the character is not considered captured. [12.83]
%
%If a character is captured and combat is still being resolved, the captured character contributes nothing to either side. [12.84]
%
%A player may only declare capture at the beginning of combat and if he is the attacker. [12.85]
%
%The * on the combat results table are ignored during kill type combat. [12.86]
%
%If during the action event, �Coup Mission aborted�, or the failure to complete an �Assassination� mission, there are no enemy units in the same environ, the characters are not captured. [12.87]


\end{document}
  