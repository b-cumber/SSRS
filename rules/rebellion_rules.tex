\section{Rebellion}

% status : in progress

\setcounter{rc}{0}

\begin{center}

  \begin{longtable}{| p{\first} | p{\second} | p{\third} | p{\fourth} |}
    \hline
    \textbf{\#}&
    \textbf{Rule Description}&
    \textbf{Reference(s)}&
    \textbf{Metric}
    \\ \hline
    
    \rn &
    
    A planet goes into Rebellion when a Start Rebellion mission is completed on the planet. &
    
    15.52 & 
    
    10
    
    \\ \hline
    
    \rn &
    
    If the planet is currently in rebellion or rebel controlled, ignore the event if the Rebel player is conducting the mission. &
    
    13.48 &
    
    5
    
    \\ \hline
    
    \rn &
    
    The Stop Rebellion mission can only be attempted by the Imperial Player. &
    
     13.53 &
     
     2
     
     \\ \hline
     
     \rn &
     
     The Stop Rebellion requires two mission letters to be completed. &
     
     13.53 &
     
     4
     
     \\ \hline
     
     \rn &
     
     A planet may be put into rebellion when the planet is in a state of unrest. &
     
     15.0 &
     
     6

     
     \\ \hline
     
     \rn &
     
     Rebellion starts by performing the �Start Rebellion� mission. &
     
     15.2 &
     
     7
     
     \\ \hline
     
     \rn &
     
     A rebellion may only be started on a planet in unrest. &
     
     15.2 &
     
     6
     
     \\ \hline
     
     \rn &
     
     A player may attempt the �Start Rebellion� mission even if the planet is not in �unrest�. &
     
     15.2 &
     
     6
     
     \\ \hline
     
     \rn &
     
     The imperial player may attempt to stop a Rebellion by attempting the �Stop Rebellion� mission. &
     
     15.3 &
     
     7
     
     \\ \hline
     
     \rn &
     
     The Rebel military units placed due to starting a rebellion are immediately placed when the rebellion starts (except according to 15.44), even before the �Start Rebellion� mission is resolved, for example. &
     
     15.43 &
     
     7
     
     \\ \hline
     
    \rn &
    
    Rebel military units that cannot be placed due to environ size restrictions are placed at the beginning of the first Rebel Operations Phase when they can be legally placed. &
    
    15.44 &
    
    5
    
    \\ \hline
    
    \rn &
    
    Rebel military units placed because of starting rebellion, are placed regardless of other rebel or imperial military units. &
    
    15.43 &
    
    4
    
    \\ \hline
    
    \rn &
    
    Rebel military units placed because of starting rebellion, can move regularly after being placed. &
    
    15.43 &
    
    4
    
    \\ \hline
    
    \rn &
    
    Rebel military units received when entering a state of rebellion may attack imperial military units if the rebel player wishes. &
    
    15.4,15.45 &
    
    5
    
    \\ \hline
    
    \rn &
    
    When the planet is placed into rebellion, the Imperial Player loses control of the planet. &
    
    15.4 &
    
    7
    
    \\ \hline
    
    \rn &
    
    When the planet goes into rebellion, the Rebel player receives force points for each environ on a planet equal to the Environ�s Resource Rating. &
    
    15.41 &
    
    5
    
    \\ \hline
    
    \rn &
    
    A �1-0� military unit costs one Force Point. &
    
    15.41 &
    
    3
    
    \\ \hline
    
    \rn &
    
    A �2-1� military unit costs three force points. &
    
    15.41 &
    
    3
    
    \\ \hline
    
    \rn &
    
    The force point from each environ can only be spent on units placed on that environ. &
    
    15.41 &
    
    3
    
    \\ \hline
    
    \rn &
    
    The military units from force point from each environ can only be spent on units with the same environ type as the environ placed on. &
    
    15.41 &
    
    2
    
    \\ \hline
    
    \rn &
    
    If the action card �Populace goes wild� is drawn during a mission, the resource value of that environ will be doubled if the planet is put into rebellion the current mission phase unless the planet is in the state of �Rebellion Stopped�. &
    
    15.46 &
    
    5
    
    \\ \hline
    
    \rn &
    
    If the action card �Populace goes wild� is drawn after a planet has been put into a state of rebellion that turn, the card has no effect. & 
    
    15.46 &
    
    3
    
    \\ \hline
    
    \rn &
    
    A Coup or Diplomacy mission cannot be performed on a planet in a state of Rebellion. &
    
    15.47 &
    
    5
    
    \\ \hline
    
    \rn &
    
    The loyalty marker on a planet in rebellion cannot be moved. &
    
    15.47 &
    
    3
    
    \\ \hline
    
    \rn &
    
    A planet can be placed in and out of rebellion any number of times. &
    
    15.48 &
    
    8
    
    \\ \hline
    
    \rn &
    
    If a planet is placed into a state of rebellion from rebellion stopped, the Rebel player does not receive units. &
    
    15.48 &
    
    4
    
    \\ \hline
    
    \newrule{Only �1-0� and �2-1� military units can be purchased using force points for a planet placed in a state of rebellion.}{15.42}{5}
    
        
    \newrule{Control/Rebellion has no effect on the level of a PDB.}{15.62}{2}
    
    
  \end{longtable}
\end{center}